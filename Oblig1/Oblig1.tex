\documentclass[11pt, a4paper]{article}

\usepackage[T1]{fontenc}
\usepackage[utf8]{inputenc}
\usepackage[norsk]{babel}
\usepackage{amsmath}
\usepackage{amsfonts}
\usepackage{graphicx}

\begin{document}
\begin{titlepage}

  \title{\normalsize FYS3150 Computational Physics 2014\\
  \vspace{10mm}
  \huge Oblig 1\\
  \vspace{10mm}
  \normalsize {\bf Løysning av lineære likningar på matriseform.}}

  \author{Øyvind Sigmundson Schøyen}

\end{titlepage}
\maketitle

\newpage
  \tableofcontents
\newpage

\section{Introduksjon}
  I dette prosjektet har me tatt for oss ein ein-dimensjonal Poisson likning med Dirichlet randpunkt.
  Me vil simulere ein numerisk løysning på ein andre-ordens differensiallikning.
  Måten me vil gjer dette på er ved å omforme settet med lineære likningar til ei matriselikning. Då 
  kan me bruke radoperasjoner til å lage ein algoritme som gjer det mogleg for oss å løyse 
  likningssettet. Denne matriselikninga vil gje oss ei tridiagonalmatrise $A$ som stort sett består av  nullar. Hensikta er då å sjå at me kan "kaste" alle nullane og kun behalde elementa frå diagonalane.
  Me vil representere desse som vektorar for å bruke minst mogleg plass. Dette vil og gjere det mogleg  for oss å kunne utføre fleire iterasjoner. Ein vanleg PC vil ikkje kunne klare å representere ein 
  stort større matrise enn $1000\times1000$. Dette fordi han ikkje har stort nok minne. Viss me 
  derimot bruker vektorar vil me kunne ha ei "matrise" (tre vektorar) som er mykje større. I tillegg 
  vil radreduksjonen gå fleirfoldige mange gonger kjappare.


\section{Omforming av differensiallikning til ei matriselikning}
  Me er interesserte i å forme om uttrykket for tilnærminga til den andrederiverte. Me startar med å 
  fjerne brøken. Då får me
  \begin{equation*}
    -\frac{v_{i+1} + v_{i-1} - 2v_i}{h^2} = f_i \ \ \ \ \Rightarrow \ \ \
    -v_{i+1} + v_{i-1} - 2v_i = h^2f_i = \tilde{b}_i, \ \forall \ i \in [1, n].
  \end{equation*}
  Me er no interesserte i å skrive dette om til vektorar og ei matrise. Me kan då skrive det som
  \begin{equation*}
    \begin{align}
      &(2, -1)\cdot(v_1, v_2) = \tilde{b}_1, \\
      &(-1, 2, -1)\cdot(v_1, v_2, v_3) = \tilde{b}_2, \\
      &(-1, 2, -1)\cdot(v_2, v_3, v_4) = \tilde{b}_3, \\
      &\dots \\
      &(-1, 2, -1)\cdot(-v_{i+1}, -v_{i-1}, v_i) = \tilde{b}_i, \\
      &\dots \\
      &(-1, 2)\cdot(v_{n-1}, v_{n}) = \tilde{b}_n.
    \end{align}
  \end{equation*}
  Det kjem fram frå dette uttrykket at koeffisientane står i ro. Viss me no setter koeffisientane i ei
  matrise vil me kun trenge ein vektor $\mathbf{v}$ beståande av alle $v_i, \ \forall \ i \in [1, n]$
  kor dei vil få dei rette koeffisientane i eit matriseprodukt. Me setter opp matriselikninga.
  \begin{equation*}
    A\mathbf{v} = 
    \begin{pmatrix}
      2 & -1 & 0 & \dots & \dots & 0 \\
      -1 & 2 & -1 & 0 & \dots & \dots \\
      0 & -1 & 2 & -1 & 0 & \dots \\
      \dots & \dots & \dots & \dots & \dots & \dots \\
      0 & \dots & \dots & -1 & 2 & -1 \\
      0 & \dots & \dots & 0 & -1 & 2
    \end{pmatrix}
    \begin{pmatrix}
      v_1 \\
      v_2 \\
      v_3 \\
      \dots \\
      v_{n-1} \\
      v_{n}
    \end{pmatrix} = 
    \begin{pmatrix}
      \tilde{b}_1 \\
      \tilde{b}_2 \\
      \tilde{b}_3 \\       
      \dots \\
      \tilde{b}_{n-1} \\
      \tilde{b}_n
    \end{pmatrix} = \mathbf{\tilde{b}}
  \end{equation*}
  Eit matriseprodukt av dette uttrykket vil gje oss
  \begin{equation*}
    \begin{align}
      &2v_1 - v_2 = \tilde{b}_1 \\
      &-v_1 + 2v_2 - v_3 = \tilde{b}_2 \\
      &\dots \\
      &-v_{i-1} + 2v_i - v_{i+1} = \tilde{b}_i \\
      &\dots \\
      &-v_{n-2} + 2v_{n-1} -v_{n} = \tilde{b}_{n-1} \\
      &-v_{n-1} + 2v_{n} = \tilde{b}_n.
    \end{align}
  \end{equation*}
  Dette fordi koeffisientane er omringa av nullar som automatisk vil fjerne resten av ledda frå 
  $\mathbf{v}$.

\section{Algoritmen}
  For å løyse ei matriselikning bruker me radoperasjoner på matrise. Desse vil igjen bli utførde på
  løysningsvektoren (her kalla $\mathbf{\tilde{b}}$). Målet med likninga er å omforme matrisa til ei spesiell øvre-  og nedrematrise. I den fyrste matrisa vil me fjerne alt som er under diagonalen medan me i den andre
  vil fjerne alt over diagonalen samt omforme diagonalen slik han utelukkande består av einarar.

\end{document}
