\documentclass[11pt, a4paper]{article}

\usepackage[T1]{fontenc}
\usepackage[utf8]{inputenc}
\usepackage[norsk]{babel}
\usepackage{amsmath}
\usepackage{amsfonts}
\usepackage{graphicx}

\begin{document}
\begin{titlepage}

  \title{\normalsize FYS3150 Computational Physics 2014\\
  \vspace{10mm}
  \huge Oblig 1\\
  \vspace{10mm}
  \normalsize {\bf Løysning av lineære likningar på matriseform.}}

  \author{Øyvind Sigmundson Schøyen}

\end{titlepage}
\maketitle

\newpage
  \tableofcontents
\newpage

\section{Introduksjon}
  I dette prosjektet har me tatt for oss ein ein-dimensjonal Poisson likning med Dirichlet randpunkt.
  Me vil simulere ein numerisk løysning på ein andre-ordens differensiallikning.
  Måten me vil gjer dette på er ved å omforme settet med lineære likningar til ei matriselikning. Då 
  kan me bruke radoperasjoner til å lage ein algoritme som gjer det mogleg for oss å løyse 
  likningssettet. Denne matriselikninga vil gje oss ei tridiagonalmatrise $A$ som stort sett består av  nullar. Hensikta er då å sjå at me kan "kaste" alle nullane og kun behalde elementa frå diagonalane.
  Me vil representere desse som vektorar for å bruke minst mogleg plass. Dette vil og gjere det mogleg  for oss å kunne utføre fleire iterasjoner. Ein vanleg PC vil ikkje kunne klare å representere ein 
  stort større matrise enn $1000\times1000$. Dette fordi han ikkje har stort nok minne. Viss me 
  derimot bruker vektorar vil me kunne ha ei "matrise" (tre vektorar) som er mykje større. I tillegg 
  vil radreduksjonen gå fleirfoldige mange gonger kjappare.


\section{Omforming av differensiallikning til ei matriselikning}
  Me er interesserte i å forme om uttrykket for tilnærminga til den andrederiverte. Me startar med å 
  fjerne brøken. Då får me
  \begin{equation*}
    -\frac{v_{i+1} + v_{i-1} - 2v_i}{h^2} = f_i \ \ \ \ \Rightarrow \ \ \
    -v_{i+1} + v_{i-1} - 2v_i = h^2f_i = \tilde{b}_i, \ \forall \ i \in [1, n].
  \end{equation*}
  Me er no interesserte i å skrive dette om til vektorar og ei matrise. Me kan då skrive det som
  \begin{equation*}
    \begin{align}
      &(2, -1)\cdot(v_1, v_2) = \tilde{b}_1, \\
      &(-1, 2, -1)\cdot(v_1, v_2, v_3) = \tilde{b}_2, \\
      &(-1, 2, -1)\cdot(v_2, v_3, v_4) = \tilde{b}_3, \\
      &\dots \\
      &(-1, 2, -1)\cdot(-v_{i+1}, -v_{i-1}, v_i) = \tilde{b}_i, \\
      &\dots \\
      &(-1, 2)\cdot(v_{n-1}, v_{n}) = \tilde{b}_n.
    \end{align}
  \end{equation*}
  Det kjem fram frå dette uttrykket at koeffisientane står i ro. Viss me no setter koeffisientane i ei
  matrise vil me kun trenge ein vektor $\mathbf{v}$ beståande av alle $v_i, \ \forall \ i \in [1, n]$
  kor dei vil få dei rette koeffisientane i eit matriseprodukt. Me setter opp matriselikninga.
  \begin{equation*}
    A\mathbf{v} = 
    \begin{pmatrix}
      2 & -1 & 0 & \dots & \dots & 0 \\
      -1 & 2 & -1 & 0 & \dots & \dots \\
      0 & -1 & 2 & -1 & 0 & \dots \\
      \dots & \dots & \dots & \dots & \dots & \dots \\
      0 & \dots & \dots & -1 & 2 & -1 \\
      0 & \dots & \dots & 0 & -1 & 2
    \end{pmatrix}
    \begin{pmatrix}
      v_1 \\
      v_2 \\
      v_3 \\
      \dots \\
      v_{n-1} \\
      v_{n}
    \end{pmatrix} = 
    \begin{pmatrix}
      \tilde{b}_1 \\
      \tilde{b}_2 \\
      \tilde{b}_3 \\       
      \dots \\
      \tilde{b}_{n-1} \\
      \tilde{b}_n
    \end{pmatrix} = \mathbf{\tilde{b}}
  \end{equation*}
  Eit matriseprodukt av dette uttrykket vil gje oss
  \begin{equation*}
    \begin{align}
      &2v_1 - v_2 = \tilde{b}_1 \\
      &-v_1 + 2v_2 - v_3 = \tilde{b}_2 \\
      &\dots \\
      &-v_{i-1} + 2v_i - v_{i+1} = \tilde{b}_i \\
      &\dots \\
      &-v_{n-2} + 2v_{n-1} -v_{n} = \tilde{b}_{n-1} \\
      &-v_{n-1} + 2v_{n} = \tilde{b}_n.
    \end{align}
  \end{equation*}
  Dette fordi koeffisientane er omringa av nullar som automatisk vil fjerne resten av ledda frå 
  $\mathbf{v}$.

  \subsection{Utleiing av eksakt løysning}
    For å sjå at den eksakte løysninga er riktig treng me berre å derivere uttrykket to gonger for å
    sjå at det vert det same som uttrykket til $f$.
    \begin{equation*}
      \begin{align}
        &u(x) = 1 - x - xe^{-10} - e^{-10x}, \\
        &\frac{du(x)}{dx} = -1 -e^{-10} + 10e^{-10x}, \\
        &\frac{d^2u(x)}{dx^2} = -100e^{-10x} = f(x).
      \end{align}
    \end{equation*}

\section{Algoritmen}
  For å løyse ei matriselikning bruker me radoperasjoner på matrisa. Desse vil igjen bli utførde på
  løysningsvektoren (her kalla $\mathbf{\tilde{b}}$). Målet med likninga er å omforme matrisa til ei 
  spesiell øvre-  og nedrematrise som me kan løyse. 
  
  \subsection{Dekomponering}
    I den fyrste matrisa vil me fjerne alle verdiar over diagonalen slik at me står igjen med $\lambda_i$ og $a_i, \ \forall \ i \in [1, n]$. I den andre vil me einarar på diagonalen og $\gamma_i, \ \forall \ i \in [1, n - 1]$ over.

    \begin{equation*}
      \begin{align}
        &A =
        \begin{pmatrix}
          b_1 & c_1 & \dots & \dots & \dots & \dots \\
          a_2 & b_2 & c_2 & \dots & \dots & \dots \\
          \dots & a_3 & b_3 & c_3 & \dots & \dots \\
          \dots & \dots & \dots & \dots & \dots & \dots \\
          \dots & \dots & \dots & a_{n-1} & b_{n-1} & c_{n-1} \\
          \dots & \dots & \dots & \dots & a_n & b_n
        \end{pmatrix} = \\
        &LU =
        \begin{pmatrix}
          \lambda_1 & \dots & \dots & \dots & \dots & \dots \\
          a_2 & \lambda_2 & \dots & \dots & \dots & \dots \\
          \dots & a_3 & \lambda_3 & \dots & \dots & \dots \\
          \dots & \dots & \dots & \dots & \dots & \dots \\
          \dots & \dots & \dots & a_{n-1} & \lambda_{n-1} & \dots \\
          \dots & \dots & \dots & \dots & a_n & \lambda_n 
        \end{pmatrix}
        \begin{pmatrix}
          1 & \gamma_1  & \dots & \dots & \dots & \dots \\
          \dots & 1 & \lambda_2 & \dots & \dots & \dots \\
          \dots & \dots & 1 & \gamma_3 & \dots & \dots \\
          \dots & \dots & \dots & \dots & \dots & \dots \\
          \dots & \dots & \dots & \dots & 1 & \gamma_{n-1} \\
          \dots & \dots & \dots & \dots & \dots & 1 
        \end{pmatrix}.
      \end{align}
    \end{equation*}
    Multiplikasjon av matrisene $LU$ vil gje oss $A$. Me bruker dette til å finne eit uttrykk for 
    $\gamma_i, \ \forall \ i \in [2, n]$ og $\lambda_i, \ \forall \ i \ \in [1, n]$. Det vil gje oss
    \begin{equation*}
      \begin{align}
        &\lambda_1 = b_1, \\
        &\lambda_1\gamma_2 = a_2, \\
        &c_1\gamma_2 + \lambda_2 = b_2, \\
        &\dots \\
        &\gamma_i = \frac{a_{i}}{\lambda_{i-1}}, \\
        &\lambda_i = b_i - c_{i-1}\gamma_i, \\
        &\dots \\
        &\gamma_n = \frac{a_n}{\lambda_{n-1}}, \\
        &\lambda_n = b_n - c_{n-1}\gamma_n.
      \end{align}
    \end{equation*}
  
\end{document}
