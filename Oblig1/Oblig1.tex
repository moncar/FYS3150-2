\documentclass[11pt, a4paper]{article}

\usepackage[T1]{fontenc}
\usepackage[utf8]{inputenc}
\usepackage[norsk]{babel}
\usepackage{amsmath}
\usepackage{amsfonts}
\usepackage{graphicx}

\begin{document}
\begin{titlepage}

  \title{\normalsize FYS3150 Computational Physics 2014\\
  \vspace{10mm}
  \huge Oblig 1\\
  \vspace{10mm}
  \normalsize {\bf Løysning av lineære likningar på matriseform.}}

  \author{Øyvind Sigmundson Schøyen}

\end{titlepage}
\maketitle

\newpage
  \tableofcontents
\newpage

\section{Introduksjon}
  I dette prosjektet har me tatt for oss ein ein-dimensjonal Poisson likning med Dirichlet randpunkt.
  Me vil simulere ein numerisk løysning på ein andre-ordens differensiallikning.
  Måten me vil gjer dette på er ved å omforme settet med lineære likningar til ei matriselikning. Då 
  kan me bruke radoperasjoner til å lage ein algoritme som gjer det mogleg for oss å løyse 
  likningssettet. Denne matriselikninga vil gje oss ei tridiagonalmatrise $A$ som stort sett består av  nullar. Hensikta er då å sjå at me kan "kaste" alle nullane og kun behalde elementa frå diagonalane.
  Me vil representere desse som vektorar for å bruke minst mogleg plass. Dette vil og gjere det mogleg  for oss å kunne utføre fleire iterasjoner. Ein vanleg PC vil ikkje kunne klare å representere ein 
  stort større matrise enn $1000\times1000$. Dette fordi han ikkje har stort nok minne. Viss me 
  derimot bruker vektorar vil me kunne ha ei "matrise" (tre vektorar) som er mykje større. I tillegg 
  vil radreduksjonen gå fleirfoldige mange gonger kjappare.

\section{Algoritmen}
  For å løyse ei matriselikning bruker me radoperasjoner på matrise. Desse vil igjen bli utførde på
  løysningsvektoren (her kalla $b^~$). 

\end{document}
