\documentclass[11pt, a4paper]{article}

\usepackage[T1]{fontenc}
\usepackage[utf8]{inputenc}
\usepackage[norsk]{babel}
\usepackage{amsmath}
\usepackage{amsfonts}
\usepackage{amssymb}
\usepackage{graphicx}
\usepackage{verbatim}
\usepackage{caption}
\usepackage{subcaption}
\usepackage{subfig}
\usepackage{float}
\usepackage{program}


\begin{document}
\begin{titlepage}

  \title{\normalsize FYS3150 Computational Physics 2014\\
  \vspace{10mm}
  \huge Oblig 2\\
  \vspace{10mm}
  \normalsize {\bf Løysning av Schrödinger likninga for to elektron i ein tredimensjonal oscillator brønn.}}

  \author{Øyvind Sigmundson Schøyen}

\end{titlepage}

\maketitle

\newpage
\begin{abstract}
  Skriv eit samandrag av prosjektet her oppe. \\
  \texttt{https://github.com/Schoyen/FYS3150/tree/master/Oblig2}
\end{abstract}

\newpage
  \tableofcontents
\newpage

\section{Introduksjon}
  I dette prosjektet er me interesserte i å løyse Schrödinger likninga for to elektron. Likninga er skrive om slik at me kan jobbe med eit ein-lekam problem 
  istadenfor to. Me nyttar lineær algebra for å løyse differensiallikningane som eit sett med lineær likningar. Måten me gjer dette på er ved Jacobirotasjon 
  for å finne eigenvektorar og eigenverdiar. Til slutt vil me plotte bølgjefunksjonen for grunntilstanden til elektrona ved hjelp av eigenvektorane og 
  eigenverdiane.


\section{Jacobirotasjon}
  For å løyse eigenverdi- og eigenvektorproblem vil me nytte Jacobirotasjon. Dette er ein algoritme som, etter ein rekke similaritetsformasjonar, vil 
  gjere alle ikkje-diagonale matriseelement til null. Denne algoritmen er i midlertid ikkje ein veldig effektiv algoritme då me ved ein rotasjon kan kome 
  i skade for å gjere eit element som tidligare var null til å bli ikkje-null. Numerisk kan det og ta lang tid før elementa vert null. Me vil difor heile 
  tida teste verdiane mot ein toleranse.

  \subsection{Algoritma}
    Ein similaritetstransformasjon er gitt ved 
    \begin{align*}
      B = S^TAS
    \end{align*}
    kor $S$ er ein ortogonal matrise der $SS^T = SS^{-1} = I$. Matrisa $S$ transformerer $A$ ein vinkel $\theta$ i planet medan $S^T$ tek ho tilbake.
    Me vil då velje $\theta$ slik at alle ikkje-diagonale element vert null. Når me gjer dette numerisk må me gjere ein rekke similaritetstransformasjonar 
    for å oppnå dette. Då har me
    \begin{align*}
      B = S_n^T \dots S_1^TAS_1 \dots S_n.
    \end{align*}
    Kvar matrise $S$ og $S^T$ er identitetsmatrisa med unntak av elementa $s_{kk} = s_{ll} = \cos{\theta}$, $s_{kl} = -s_{lk} = -\sin{\theta}$ og 
    $s_{ii} = 1$ for $i \ne k$ og $i \ne l$. Produktet $B = S^TAS$ kan då skrivast som
    \begin{align*}
      b_{ii} &= a_{ii}, \qquad i \ne k, \ i \ne l \\
      b_{ik} &= a_{ik}\cos{\theta} - a_{il}\sin{\theta}, \qquad i \ne k, \ i \ne l \\
      b_{il} &= a_{il}\cos{\theta} + a_{ik}\sin{\theta}, \qquad i \ne k, \ i \ne l \\
      b_{kk} &= a_{kk}\cos^2{\theta} - 2a_{kl}\cos{\theta}\sin{\theta} + a_{ll}\sin^2{\theta} \\
      b_{ll} &= a_{ll}\cos^2{\theta} + 2a_{kl}\cos{\theta}\sin{\theta} + a_{kk}\sin^2{\theta} \\
      b_{kl} &= (a_{kk} - a_{ll})\cos{\theta}\sin{\theta} + a_{kl}(\cos^2{\theta} - \sin^2{\theta}).
    \end{align*}
    Me vil no velje $\theta$ slik at alle ikkje-diagonale element $b_{kl}$ i praksis vert null. For kvar iterasjon vil me då teste om summen av alle 
    dei ikkje-diagonale elementa er mindre enn ein toleranse $\epsilon$. Me vil i midlertid ikkje gjer dette då det er ein tidkrevande prosess. 
    Erstatninga vert då å sjå om det største elementet blant dei ikkje-diagonale elementa er mindre enn $\epsilon$.

\end{document}
