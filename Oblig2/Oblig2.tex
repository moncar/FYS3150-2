\documentclass[11pt, a4paper]{article}

\usepackage[T1]{fontenc}
\usepackage[utf8]{inputenc}
\usepackage[norsk]{babel}
\usepackage{amsmath}
\usepackage{amsfonts}
\usepackage{amssymb}
\usepackage{graphicx}
\usepackage{verbatim}
\usepackage{caption}
\usepackage{subcaption}
\usepackage{subfig}
\usepackage{float}
\usepackage{program}
\usepackage{commath}


\begin{document}
\begin{titlepage}

  \title{\normalsize FYS3150 Computational Physics 2014\\
  \vspace{10mm}
  \huge Oblig 2\\
  \vspace{10mm}
  \normalsize {\bf Løysning av Schrödinger likninga for to elektron i ein tredimensjonal oscillator brønn.}}

  \author{Øyvind Sigmundson Schøyen}

\end{titlepage}

\maketitle

\newpage
\begin{abstract}
  Skriv eit samandrag av prosjektet her oppe. \\
  \texttt{https://github.com/Schoyen/FYS3150/tree/master/Oblig2}
\end{abstract}

\newpage
  \tableofcontents
\newpage

\section{Introduksjon}
  I dette prosjektet er me interesserte i å løyse Schrödinger likninga for to elektron. Likninga er skrive om slik at me kan jobbe med eit ein-lekam problem 
  istadenfor to. Me nyttar lineær algebra for å løyse differensiallikningane som eit sett med lineær likningar. Måten me gjer dette på er ved Jacobirotasjon 
  for å finne eigenvektorar og eigenverdiar. Til slutt vil me plotte bølgjefunksjonen for grunntilstanden til elektrona ved hjelp av eigenvektorane og 
  eigenverdiane.


\section{Jacobirotasjon}
  For å løyse eigenverdi- og eigenvektorproblem vil me nytte Jacobirotasjon. Dette er ein algoritme som, etter ein rekke similaritetsformasjonar, vil 
  gjere alle ikkje-diagonale matriseelement til null. Denne algoritmen er likevel ikkje ein veldig effektiv algoritme då me ved ein rotasjon kan kome 
  i skade for å gjere eit element som tidligare var null til å bli ikkje-null. Numerisk kan det og ta lang tid før elementa vert null. Me vil difor heile 
  tida teste verdiane mot ein toleranse.

  % Det er kanskje ein idè å splitte algoritme framstillinga i ein numerisk og ein analytisk del?
  \subsection{Algoritma}
    Ein similaritetstransformasjon er gitt ved 
    \begin{align*}
      B = S^TAS
    \end{align*}
    kor $S$ er ein ortogonal matrise der $SS^T = SS^{-1} = I$. Matrisa $S$ transformerer $A$ ein vinkel $\theta$ i planet medan $S^T$ tek ho tilbake.
    Me vil då velje $\theta$ slik at alle ikkje-diagonale element vert null. Når me gjer dette numerisk må me gjere ein rekke similaritetstransformasjonar 
    for å oppnå dette. Då har me
    \begin{align*}
      B = S_n^T \dots S_1^TAS_1 \dots S_n.
    \end{align*}
    Kvar matrise $S$ og $S^T$ er identitetsmatrisa med unntak av elementa $s_{kk} = s_{ll} = \cos{\theta}$, $s_{kl} = -s_{lk} = -\sin{\theta}$ og 
    $s_{ii} = 1$ for $i \ne k$ og $i \ne l$. Produktet $B = S^TAS$ kan då skrivast som
    \begin{align*}
      b_{ii} &= a_{ii}, \qquad i \ne k, \ i \ne l \\
      b_{ik} &= a_{ik}\cos{\theta} - a_{il}\sin{\theta}, \qquad i \ne k, \ i \ne l \\
      b_{il} &= a_{il}\cos{\theta} + a_{ik}\sin{\theta}, \qquad i \ne k, \ i \ne l \\
      b_{kk} &= a_{kk}\cos^2{\theta} - 2a_{kl}\cos{\theta}\sin{\theta} + a_{ll}\sin^2{\theta} \\
      b_{ll} &= a_{ll}\cos^2{\theta} + 2a_{kl}\cos{\theta}\sin{\theta} + a_{kk}\sin^2{\theta} \\
      b_{kl} &= (a_{kk} - a_{ll})\cos{\theta}\sin{\theta} + a_{kl}(\cos^2{\theta} - \sin^2{\theta}).
    \end{align*}
    Me vil no velje $\theta$ slik at alle ikkje-diagonale element $b_{kl}$ i praksis vert null. For kvar iterasjon vil me då teste om summen av alle 
    dei ikkje-diagonale elementa er mindre enn ein toleranse $\epsilon$. Me vil derimot ikkje gjer dette då det er ein tidkrevande prosess. 
    Erstatninga vert då å sjå om det største elementet blant dei ikkje-diagonale elementa er mindre enn $\epsilon$. Dette vil då vere
    \begin{align*}
      \abs{a_{kl}} = \text{max}_{i \ne j}\abs{a_{ij}}.
    \end{align*}
    Me krever at $b_{kl} = b_{lk} = 0$. Det gjer oss likninga
    \begin{align}
      a_{kl}(c^2 - s^2) + (a_{kk} - a_{ll})cs = b_{kl} = 0.
    \end{align}
    Me definerer no 
    \begin{align*}
      \tau = \frac{a_{ll} - a_{kk}}{2a_{kl}} \qquad \Rightarrow \qquad a_{ll} - a_{kk} = 2\tau a_{kl}.
    \end{align*}
    Setter dette inn i (1) og får
    \begin{align*}
      &a_{kl}c^2 - a_{kl}s^2 - 2\tau a_{kl}cs = 0, \\
      \Rightarrow \qquad &a_{kl} - a_{kl}\frac{s^2}{c^2} - 2\tau a_{kl}\frac{s}{c} = 0. \\
    \end{align*}
    Siden $c = \cos{\theta}$ og $s = \sin{\theta}$ vil me få 
    \begin{align*}
      &a_{kl} - a_{kl}\tan^2{\theta} - 2\tau a_{kl} \tan{\theta} = 0, \\
      \Rightarrow \qquad &1 - t^2 - 2\tau t = 0, \\
      \Rightarrow \qquad &t^2 + 2\tau t - 1 = 0,
    \end{align*}
    kor $t = \tan{\theta}$. Me vil då få 
    \begin{align*}
      t &= \frac{-2\tau \pm \sqrt{4\tau^2 - 4(-1)}}{2} \\
      &= -\tau \pm \sqrt{1 + \tau^2}.
    \end{align*}
    For å unngå avrungdingsfeil ved $\tau >> 0$ gonger me andregradslikninga med den konjugerte. Det vil gje oss
    \begin{align*}
      t &= \left( -\tau \pm \sqrt{1 + \tau^2} \right)\left( \frac{-\tau \pm \sqrt{1 + \tau^2}}{-\tau \pm \sqrt{1 + \tau^2}} \right) \\
      &= \frac{-\tau^2 + (1 + \tau^2)}{\tau \pm \sqrt{1 + \tau^2}} = \frac{1}{\tau \pm \sqrt{1 + \tau^2}},
    \end{align*}
    som gjer oss
    \begin{align*}
      t = \frac{1}{\tau \pm \sqrt{1 + \tau^2}} \qquad \lor \qquad t = \frac{-1}{-\tau + \sqrt{1 + \tau^2}}.
    \end{align*}
    Me vil no velje den minste av røttene $t$ slik at $c$ og $s$ henholdsvis går mot ein og null ved
    \begin{align*}
      c = \frac{1}{\sqrt{1 + t^2}}, \qquad s = tc.
    \end{align*}
    Då vil me kunne begrense vinkelen $\theta \leq \frac{\pi}{4}$ slik at differansen mellom matrisa $B$ og $A$ vert så liten som mogleg ved formelen
    \begin{align*}
      \lvert B - A \rvert_F^2 = 4\underbrace{(1 - c)}_{= 0 \text{ for } c \to 1}\sum_{i = 1, i\ne k, l}^n(a_{ik}^2 + a_{il}^2) + \frac{2a_{kl}^2}{c^2}.
    \end{align*}
    Me vil fortsette desse operasjonane heilt til $\text{max}(a_{ij})^2 \leq \epsilon$ for $i \ne j$.

  \subsection{Implementering}
    Algoritma vert implementert i ein klasse med tre metodar som løyser eigenverdi og eigenvektor problemet.
    % Vert det nødvendig med pseudokode?
    % Ta med implementeringa av eigenvektorproblemet.



\section{Resultat}
  Denne seksjonen vil bli delt opp i fire deler. Me vil diskutere metoden i seg sjølv kor me ser på køyretid og stabilitet. % Big O, Jacobi mot 
  % Armadillo og storleik på matrise.
  Me vil sjå på energinivåa til elektrona med, og uten, Coulomb interaksjon for forskjellige $\omega_r$ og $\rho_{\text{max}}$. Eg har valt å gjere dette 
  ved Jacobi. Dette krev at me normaliserer eigenvektorane for å få riktige verdiar. Det vil bli forklart i meir nøyaktighet. % FORKLAR!
  Til slutt har eg lagt ved ein feilanalyse på eigenverdiane som funksjon av $n_{\text{step}}$.

  \subsection{Jacobi}
    Jacobirotasjon er ein reknetung algoritme. Denne metoda bryt veldig kjapt ned når ein lager ei stor matrise. For vårt formål er difor 
    essensielt at me uttnyttar peikar- og referansemoglegheitane i C++ for å unngå unødvendig tidbruk på lagring. Køyretida til metoden går som $\mathcal{O}(n^3)$ for rotasjon. 
    Mykje av grunnen til at metoda er treig kjem frå det faktum at me står i fare for å gjere eit element som er null til ikkje-null ved ein rotasjon. Me merker fort når matrisa 
    blir stor at tidbruken stig veldig kjapt. Ein samanlikning av $\texttt{eig\_sym}$ frå Armadillo og Jacobi gjer oss køyretidene i sekund i tabellen under.
    \begin{center}
      \begin{tabular}{|l||l|l|}
        \hline
        Steg & \text{Armadillo} & \text{Jacobi} \\
        \hline
        \input{koyretid.txt} \\
        \hline
      \end{tabular}
    \end{center}
    Me ser korleis Armadillo aukar jamnt med $n$-verdiane medan Jacobi stig eksponensielt. Med ei kubisk køyretid vil me veldig kjapt nå eit tak over kor stor matrisa vår kan 
    vere. I programmet vårt set me ei begrensning for kor mange iterasjonar me tillet. Når matrisa vert stor nok vil denne begrensninga ikkje lengre tjene formålet då ho vil 
    verte for stor for våre praktiske formål. Ei senkning av denne kan igjen føre til eit for lite antal iterasjoner og gå på bekostning av presisjon. Metoden i seg sjølv 
    er difor stabil for mindre matrisar, men kan fort bli ubrukeleg for større matrisar. Fordelane med metoda er at ho er enkel å implementere og er effektiv ved 
    parallellprogrammering. Me kjem diverre ikkje inn på dette i denne oppgåva.

  \subsection{Potensialenergien til elektron med og uten Coulomb interaksjon}

    

\end{document}
